\documentclass{ltjsarticle}
\usepackage{amsmath}
\usepackage{url}
\usepackage{listings,jvlisting}
\lstset{
  basicstyle={\ttfamily},
  identifierstyle={\small},
  commentstyle={\small\itshape},
  keywordstyle={\small\bfseries},
  ndkeywordstyle={\small},
  stringstyle={\small\ttfamily},
  frame={tb},
  breaklines=true,
  columns=[l]{fullflexible},
  numbers=left,
  xrightmargin=0\zw,
  xleftmargin=3\zw,
  numberstyle={\scriptsize},
  stepnumber=1,
  numbersep=1\zw,
  lineskip=-0.5ex
}
\begin{document}
\section{補足情報}
ラズパイのユーザ名、パスワード、ホスト名はすべて``tbw''.
\section{追加で行ったこと}
\subsection{VNC接続}
\verb|sudo raspi-config|中で
3. Interface Options $\to$ I3. VNC $\to$ enable
としてVNCを有効化する。そして、\verb|/etc/wayvnc/config|の2--3行目を以下のように書き換える。

\begin{lstlisting}[caption={/etc/wayvnc/config (変更前)}]
use_relative_paths=true
address=::
enable_auth=true
enable_pam=true
private_key_file=tls_key.pem
certificate_file=tls_cert.pem
rsa_private_key_file=rsa_key.pem
\end{lstlisting}
\begin{lstlisting}[caption={/etc/wayvnc/config (変更後)}]
use_relative_paths=true
address=localhost
enable_auth=false
enable_pam=false
private_key_file=tls_key.pem
certificate_file=tls_cert.pem
rsa_private_key_file=rsa_key.pem
\end{lstlisting}
さらに、\verb|/etc/hosts|と\verb|/etc/hostname|のホスト名をtbw.localに変更する。
これで、Remminaなどからssh経由でVNC接続できる\cite{sshVNC}。


\begin{thebibliography}{9}
  \bibitem{sshVNC}``試行錯誤な日々: Raspberry Piにssh経由でVNC接続''. 2024/12/22. \url{https://asukiaaa.blogspot.com/2024/12/vnc-connect-to-raspberry-pi-via-ssh-tunnel.html} (2025/9/6)
\end{thebibliography}
\end{document}