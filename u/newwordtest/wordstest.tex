\documentclass[a4paper]{ltjsarticle}
\usepackage{./files/mywordstest}
\usepackage[margin=15mm]{geometry}
\usepackage{luatexja-ruby}
 \renewcommand{\anscolor}{red}
\pagestyle{empty}
\begin{document}
\leftskip=1em
\TitleHead[aaa]{1}

\section{\blank に当てはまる適切な単語または表現を選択肢の中から一つ選びなさい。}
\JtoE{Cultive \blank}{小麦を栽培する}{walnut}{wool}{wheat}{wreck}{3}{}{1042}
\JtoE{\blank deep into the jungle}{ジャングルに奥深く入り込む}{penetrate}{preach}{press}{punish}{1}{}{1238}
\JtoE{He \blank the position of Chemical King.}{彼は化学王の地位に適任だ。}{qualifies to}{qualifies for}{quacks to}{quacks for}{2}{}{645}
\JtoE{class in \blank}{混沌とした授業}{cliff}{chaos}{civil}{choke}{2}{}{1488}
\JtoE{He's popular with his \blank}{彼は同僚に人気だ}{peer}{peel}{peek}{peen}{1}{}{784}
\JtoE{the \blank for ``Pi Never Ends''}{「円周率は終わらない」の台本}{sewage}{shelf}{scorn}{script}{4}{}{1524}
\JtoE{incomprehensible \blank}{理解不能な文脈}{tex}{latex}{context}{word}{3}{}{738}
\JtoE{\blank the promise}{約束を果たす}{fulfill}{fund}{fury}{frighten}{1}{}{635}
\JtoE{\blank like orca}{シャチのような捕食動物}{predecessors}{pregnants}{predetors}{prescriptions}{3}{}{1572}
\JtoE{\blank laughter}{自然に起こる笑い}{splendid}{spectacler}{subordinate}{spontaneous}{4}{}{1677}

\section{下線部の単語または表現を和訳しなさい。}
\EtoJ{schedule of \underline{surgery}}{手術}{1097}
\EtoJ{He likes to use \underline{proverbs}.}{ことわざ}{1467}
\EtoJ{have no \underline{appetite}}{食欲}{1003}
\EtoJ{\underline{devorce} registration}{離婚}{748}
\EtoJ{\underline{inhabitants} of another world}{住民}{1015}
\EtoJ{The Ice \underline{Castle}}{城}{1044}
\EtoJ{Enjoy \underline{prison} life to the fullest}{刑務所}{979}
\EtoJ{traditional \underline{remedies} for the disease}{治療法}{1479}
\EtoJ{oppose the \underline{pension} system}{年金}{1525}
\EtoJ{live in \underline{poverty}}{貧乏}{932}
\EtoJ{\underline{dissolve} metals in chemicals}{溶かす}{1223}
\EtoJ{\underline{trial version} of the game}{体験版}{731}
\EtoJ{\underline{certificate} of lateness}{証明書}{1545}
\EtoJ{\underline{confront} bosses' domination}{立ち向かう}{624}
\EtoJ{\underline{cope well with} the baby}{うまく対処する}{861}

\section{次の各文の\blank に最もよく当てはまる語または表現を答えなさい。}
\EW{The dangers \blank[i]\blank[i] the experiment}{その実験に元から伴う危険}{inherent in}{1681}
\EW{a long \blank[s] of cloth}{細い布切れ}{strip}{770}
\EW{\blank[b] the magnet}{磁石を埋める}{bury}{860}
\EW{the \blank[m] of conflict}{紛争の悲惨さ}{misery}{1462}
\EW{work for a \blank[w] of 35 yen per hour}{時給35円の賃金で働く}{wage}{936}

{\color{\anscolor} 大問2の和訳がほしい方は気軽にお申し付けください}

\end{document}