\documentclass[landscape,a5paper]{ltjsarticle}
\usepackage{luatexja-preset}
\pagestyle{empty}

\begin{document}
\begin{center}
\textbf{\Large\gtfamily\sffamily 2025年度 <関 正生の TOEIC\textregistered\ L\&Rテスト 神単語 準拠 試験範囲表>}\\
\end{center}
\begin{enumerate}
  \item 配信日:以下の日程通り.やる気等の関係で前後する可能性が高い.
  \item テストの成績は英語コミュニケーションI\!Vの評価に組み込まれない.
  \item \underline{未受験は0点扱い.公欠・忌引きはカウントしない.(自分で申し出ること)}
\end{enumerate}
\begin{table}[h]
  \centering
  \begin{tabular}{|p{3em}|p{14em}|p{14em}|p{4em}|}
    \hline
    \centering 回数&日付&\centering 単語番号 範囲&問題数\\\hline
    \centering\bfseries\sffamily 1  & 6月21日(土)&  1--200 &\hfill 30\\\hline
    \centering\bfseries\sffamily 2  & 7月 5日(土)&201--400 &\hfill 30\\\hline
    \centering\bfseries\sffamily 3  & 7月19日(土)&401--600 &\hfill 30\\\hline
    \centering\bfseries\sffamily 4  & 8月 2日(土)&601--800 &\hfill 30\\\hline
    \centering\bfseries\sffamily5\textasciitilde&約2週毎&全部&\hfill 30\\\hline  
  \end{tabular}
\end{table}

% \vfill
\begin{itemize}
  \item \underline{不正行為や疑わしい行為は,自己判断で行うこと.}
  \item \underline{テストのときは,机上に鉛筆・シャープペンシル・消しゴム・赤ペン・スマホ・パソコン・教科書・キムワイプ・ラムネ等のみ.}
  \item 不合格とは,\underline{そんなものは存在しない.}
  \item 間違えた単語は,よく復習すること.
  \item テストの解答は,何かしらの方法で採点者に送信すること.
  \item テストの内容は\underline{無断で}第三者に送信しないこと.
\end{itemize}

\end{document}